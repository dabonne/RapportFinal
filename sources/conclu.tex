%------------------------------%
%------------------------------%
\chapter*{Conclusion Générale}
%------------------------------%
%------------------------------%
%\thispagestyle{plain}
\addcontentsline{toc}{chapter}{Conclusion générale}
\markboth{Conclusion}{Conclusion}

Le problème qui nous a été posé était l'application du machine learning dans 
l'analyse des opérations à l'étranger. De nombreuses opérations sont menées 
quotidiennement par la Société Générale Burkina Faso. Ces opérations sont 
délicates car engageant de nombreuses personnes: la personne qui émet 
l'opération, son correspondant bénéficiaire de l'opération, la banque émettrice
 et celle bénéficiaire et les nombreux intermédiaires.


L'expérience que nous avons mené durant nos 6 mois de stage à la Société Générale,
nous ont permis de comprendre le processus d'analyse et de mise en conformité 
d'une opération à l'étranger. Ce stage nous a également permis de détecter les
processus d'analyse qui peuvent être automatisés grâce à l'aprentissage 
automatique.


L'objectif de notre stage était de mettre en place une plateforme intelligente qui
permettrait d'analyser la conformité des dossiers de transferts déposés au guichet
de la société Générale Burkina Faso. La bonne réalisation de notre projet nous 
contraignait à une exploration dans l'univers des algorithmes de classification
afin de trouver celui qui répondait le mieux au spécification de notre 
problème. Notre choix s'est porté sur les arbres de décisions et c'est grâce à 
eux que nous avons réalisé notre projet.
Le jeu de donnée que nous avons utilisé pour entrainer notre modèle a été obtenu à
partir des dossiers physiques et grâce à la collaboration avec les membres  
des services impliqués dans le processus d'analyse du dossier d'une opération à
 l'étranger.


Le test d’évaluation de notre modèle a donné un score de 63\%. Les résultats obtenus
 mettent en évidence l’inégale répartition de nos données dans les différentes 
 classes. En effet, le modèle prédit mieux les opérations non conformes que 
 les dossiers conformes. Une seconde approche qui visera un approfondissement de
 notre modèle ne produira-t-elle pas de meilleurs résultats?


