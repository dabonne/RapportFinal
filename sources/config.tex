

%% packages utilises
%%---------------------
%\usepackage{etex}


\usepackage[utf8x]{inputenc}
\usepackage[T1]{fontenc}
\usepackage[english, french]{babel}
\usepackage[french]{translator}
\usepackage{frbib}
\usepackage{amsmath}
\usepackage{amssymb}
\usepackage{these}

\usepackage[left=3.5cm, right=2cm, top=2cm, bottom=2cm]{geometry}

\usepackage{ulem}
%\normalem
%\usepackage{rotating}
\usepackage{tabularx}
\usepackage{textcase}
\usepackage{floatflt}
\usepackage{graphicx}
\usepackage{epstopdf}		% .eps to .pdf
\usepackage{moreverb} %% pour le verbatim en boite
\usepackage{multirow} %% pour regrouper un texte sur plusieurs lignes dans une table
\usepackage{url} %% pour citer les url par \url
\urlstyle{sf}
\usepackage[all]{xy} %% pour la barre au dessus des symboles
%\usepackage{shorttoc} %% pour plusieurs tables des matières par la commande \shorttableofcontents{Titre}{profondeur}.
\usepackage{textcomp} %% pour le symbol pour mille par \textperthousand.

%\usepackage[right]{eurosym}

\usepackage{latexsym}
% 
\usepackage{setspace} %allow to change line spacing
\usepackage{fancyhdr} %header / footer
  \setlength{\headheight}{36pt}
\usepackage[ Conny ]{ fncychap }
\usepackage{enumitem} %beautiful enumerations ... no beamer
\usepackage{subfig} %allow to split figure (article / book / report)
\usepackage{caption} %
\usepackage{color} %use color
\usepackage{epsfig,amsfonts} %math mode
\usepackage{array,colortbl} %use array / tabular and apply color / style on them
\usepackage{pstricks} %draw with latex
%\usepackage{tikz}
\usepackage{listings} %source code with avec \being{lstlistgin}

\usepackage{stmaryrd}
%\usepackage{tocbibind}
%\usepackage[nottoc]{tocbibind}
\usepackage{pdfpages}
\usepackage{chngpage}
\usepackage{multicol}
%\usepackage[fit]{truncate}
\pagestyle{fancy}
\fancyhead{} %on efface l'entête

\fancyhead[L]{\includegraphics[width=1.0cm]{images/esi.jpeg}}
\fancyhead[C]{\scriptsize\textsc{Segmentation Géométrique et Photométrique d’images Acquises par Drones}}
\fancyhead[R]{\includegraphics[width=1cm]{images/liris.png}}
%\fancyhead[LE,RO]{\nouppercase{\truncate{0.5\headwidth}{\rightmark}}}
%\fancyhead[LO,RE]{\nouppercase{\truncate{0.5\headwidth}{\leftmark}}}
\fancyfoot{}
\renewcommand{\footrulewidth}{1pt}
\fancyfoot[L]{\scriptsize\textbf{DABONNE Hoda}}
\fancyfoot[C]{\includegraphics[width=1cm]{images/Technidrone.jpg}}
\fancyfoot[R]{\scriptsize\textit\thepage}

%%%%%%%%%%%%%%%%%%%style front%%%%%%%%%%%%%%%%%%%%%%%%%%%%%%%%%%%%%%%%% 
        \fancypagestyle{front}{%
                \fancyhf{}%on vide l'en-tête
                 \fancyfoot[C]{}
                \fancyfoot[R]{\scriptsize \textit\thepage}%
                \fancyfoot[L]{\scriptsize DABONNE Hoda}

                \renewcommand{\headrulewidth}{0pt}%trait horizontal pour l'en-tête
                \renewcommand{\footrulewidth}{1pt}%trait horizontal pour le pied de page
                }
%%%%%%%%%%%%%%%%%%%style main%%%%%%%%%%%%%%%%%%%%%%%%%%%%%%%%%%%%
        \fancypagestyle{*}{%
                \fancyhf{}
                %\renewcommand{\chaptermark}[1]{\markboth{\chaptername\ \thechapter.\ ##1}{}}% redéfinition pour avoir ici les titres des chapitres des sections en minuscules
                \renewcommand{\sectionmark}[1]{\markright{\thesection\ ##1}}
                \fancyhead[c]{\scriptsize\textsc{Segmentation Géométrique et Photométrique d’images Acquises par Drones}}
                \fancyhead[L]{\includegraphics[width=1.2cm]{images/esi.jpeg}}%
				\fancyhead[R]{\includegraphics[width=1cm]{images/liris.png}}
				\fancyhead[R]{\includegraphics[width=1cm]{images/Technidrone.jpg}}

                \fancyfoot[C]{}
                \fancyfoot[R]{\scriptsize \textit\thepage}%
                \fancyfoot[L]{\scriptsize DABONNE HODA}
                }
%%%%%%%%%%%%%%%%%%%style back%%%%%%%%%%%%%%%%%%%%%%%%%%%%%%%%%%%%%%%%%  
        \fancypagestyle{back}{%
                \fancyhf{}%on vide l'en-tête
                \fancyfoot[C]{page \thepage}%
                \renewcommand{\headrulewidth}{0pt}%trait horizontal pour l'en-tête
                \renewcommand{\footrulewidth}{0.4pt}%trait horizontal pour le pied de pages
                }

\usepackage[Algorithme]{algorithm}
\usepackage{algpseudocode}

\usepackage{hyperref}
\usepackage[toc,nonumberlist]{glossaries}
\usepackage{wrapfig}

\usepackage[resetlabels]{multibib}
\newcites{publis}{Publications personnelles}

%% choix des profondeurs de section pour la table des matières
%% 2= subsection, 3=subsubsection
\setcounter{secnumdepth}{2}  %% Avec un numero.
\setcounter{tocdepth}{2}     %% Visibles dans la table des matieres

\makeglossary
\def\underscore{\char`\_}

\usepackage{nomencl}
%\renewcommand{\}{Liste des notations}
%\renewcommand*{\pagedeclaration}[1]{\unskip\dotfill\hyperpage{#1}}
\makenomenclature

\usepackage{makeidx}
\makeindex


\definecolor{colKeys}{rgb}{0,0,1}
\definecolor{colIdentifier}{rgb}{0,0,0}
\definecolor{colComments}{rgb}{0,0.5,1}
\definecolor{colString}{rgb}{0.6,0.1,0.1}

\definecolor{c1}{RGB}{219,144,71}
\definecolor{c2}{RGB}{100,212,78}
\definecolor{c3}{RGB}{255,111,111}
\definecolor{c4}{RGB}{111,139,255}

\definecolor{gris}{gray}{0.45}

\lstset{%configuration de listings
	float=hbp,%
	language=C,
	basicstyle=\ttfamily\small, %
	identifierstyle=\color{colIdentifier}, %
	keywordstyle=\color{colKeys}, %
	stringstyle=\color{colString}, %
	commentstyle=\color{colComments}, %
	columns=flexible, %
	tabsize=3, %
	%frame=trBL, %
	extendedchars=true, %
	%showspaces=false, %
	showstringspaces=false, %
	numbers=none, %
	breaklines=true, %
	breakautoindent=true, %
	captionpos=b,%
	xrightmargin=0cm, %
	xleftmargin=0cm,
	mathescape=true
}


\frenchspacing

\newgeometry{hmargin={0pt,0pt},vmargin={0pt,0pt}}
\savegeometry{include}

\newgeometry{vmargin={4.1cm,3.6cm},hmargin={3cm,2cm},twoside}
\setstretch{1.1}
\savegeometry{normalpreprint}

\newgeometry{vmargin={4.1cm,3.6cm},hmargin={2cm,3cm},twoside}
\setstretch{1.1}
\savegeometry{normalpreprintinverse}

\newgeometry{vmargin={4.1cm,3.6cm},hmargin={2.5cm,2.5cm}}
\setstretch{1.1}
\savegeometry{normaldigital}

\ifthenelse{\equal{\preprint}{YES}}
{
\loadgeometry{normalpreprint}
\savegeometry{normal}
\loadgeometry{normalpreprintinverse}
\savegeometry{normalinverse}
}
{
\loadgeometry{normaldigital}
\savegeometry{normal}
\savegeometry{normalinverse}
}


%%%%%%%%%%%%%%%%%%%%%%%%%%%%%%%%%%%%%%%%%%%%%%%%%%%%%%%%%%%%%
%	SubFloat environment for lstlisting inside subloat		%
%%%%%%%%%%%%%%%%%%%%%%%%%%%%%%%%%%%%%%%%%%%%%%%%%%%%%%%%%%%%%

\captionsetup[subfloat]{
}

\makeatletter
\newbox\sf@box
\newenvironment{SubFloat}[2][]%
	{	\def\sf@one{#1}%
		\def\sf@two{#2}%
		\setbox\sf@box\hbox
			\bgroup}%
	{	\egroup
	\ifx\@empty\sf@two\@empty\relax
		\def\sf@two{\@empty}
	\fi
	\ifx\@empty\sf@one\@empty\relax
		\subfloat[\sf@two]{\box\sf@box}%
	\else
		\subfloat[\sf@one][\sf@two]{\box\sf@box}%
	\fi}
\makeatother

\usepackage{bibentry}
\nobibliography*

%% macro/racourcis por les symboles et commandes usuelles
\input{sources/macro.tex}

%%%%%%%%%%%%%%%%%%%%%%%%%%%%%%%%%%%%%%%%%%%%%%%%%%%%%%%%%%
%%%%%%%%%%%%%%%%%		Exemple		%%%%%%%%%%%%%%%%%%%%%%
%%%%%%%%%%%%%%%%%%%%%%%%%%%%%%%%%%%%%%%%%%%%%%%%%%%%%%%%%%

% glossary entry : defines the identifier in the glossary table (1rst arg), 
% and the description in second arg.
\newglossaryentry{id}{
name={ID}, %what appear in text
description={\emph{Identifier} : Identifiant}, %what appear in glossary
first={ID (\emph{Identifier} : Identifiant)}} %what appear the first time in text
\newcommand{\id}{\gls{id}} % shortcut pour the glossary entry. Simply coll \id in the text
\newcommand{\ID}{ID} % to put the text without using the glossary shortcut.


\newglossaryentry{sg}{
name={SG}, %what appear in text
description={Société Générale}, %what appear in glossary
first={SG ( Société Générale)}} %what appear the first time in text
\newcommand{\sg}{\gls{sg}} % shortcut pour the glossary entry. Simply coll \id in the text

\newglossaryentry{sgbf}{
name={SGBF}, %what appear in text
description={Société Générale Burkina Faso}, %what appear in glossary
first={SGBF ( Société Générale Burkina Faso)}} %what appear the first time in text
\newcommand{\sgbf}{\gls{sgbf}} % shortcut pour the glossary entry. Simply coll \id in the text




%%  1ere de Couverture (titre sur une seule ligne latex)
\titre{Mémoire de fin de cycle master SI-SAD}

\soutenue
%%   Laisser cette ligne en commentaire sauf pour la version finale.
%%   (la premiere page contiendra "a soutenir le ..."
%%   au lieu de "soutenue le ...")


%% Les différents champs de la couverture...
\datesout{`` fin 20xx ''}
\Auteur{Hoda}{DABONNE}%sur une seule ligne latex

\ecole{Ecole Supérieur d'Informatique}  % { rennes1 | insa | ens }
\Labo{Lab.}
\Umr{XXXX}
\LaboEtendu{Laboratoire de recherche} % sauf INSA
\ComposanteUniversitaire{Université Nazi Boni} % sauf INSA

%% La composition du jury : prénom, nom, titre
\President[Mme]{Prenom}{Nom}{Titre} %% le président du jury
\Rapporteur{Prenom}{Nom}{Titre}
\Rapporteur[Mme]{Prenom}{Nom}{Titre}
%% Si vous avez N rapporteurs ça marche toujours...
\Examinateur[Mme]{Prenom}{Nom}{Titre}
\Examinateur{Prenom}{Nom}{Titre}
%% idem...
\Advisor{Serge}{Miguet}{Maitre de Conférences}
\Advisor{Prénom}{Nom}{Titre}

\Mention{Thèmatique}
\Keywords{Mots clés}
\ordre{00000}    %% le numéro d'ordre donné par la Scol.


\title{\TITRE}

\author{\AUTp \AUTn}

\hypersetup{
	pdfauthor={{\AUTp} {\AUTn}},
	pdftitle={{\TITRE}},
	pdfsubject={\MENTION},
	pdfkeywords={\KEYWORDS},
	hidelinks
}
