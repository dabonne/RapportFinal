

\begin{table}

  \begin{center}
    \begin{scriptsize}
      \renewcommand{\arraystretch}{2}
      \begin{tabular}{|c|l|l|}
        \hline
        \rowcolor[gray]{.7}
        \bf \rule[-0.4cm]{0mm}{1cm} Section & \bf Caractéristiques & \bf Exemples\\
        \hline
        \multirow{7}{*}{\bf Emetteur de l'ordre} & Type de personne & Personne 
        physique 
        \tabularnewline
        & Identité & XXXXXX XXXXXX \tabularnewline 
        & Résultat du contrôle de l'émetteur & Aucune sanction  \tabularnewline
        & Pays de résidence & Burkina Faso \tabularnewline
        & Notation du pays de résidence & LOW \tabularnewline
        & Banque de l’émetteur & SGBF \tabularnewline
        & Pays de la banque & Burkina Faso \tabularnewline
        & Notation Pays de la banque & LOW \tabularnewline
        & Résultat du contrôle sur la banque & Aucune sanction  \tabularnewline
        & Activité de l'emetteur & Activités extractives \tabularnewline 
        \hline

        \multirow{7}{*}{\bf Bénéficiaire de l'ordre} & Type de personne & Personne
        morale 
        \tabularnewline
        & Identité & ZZZZZZZZZZ \tabularnewline
        & Résultat du contrôle sur la personne & Aucune sanction  \tabularnewline
        & Pays de résidence & France \tabularnewline 
        & Notation du pays de résidence & LOW \tabularnewline
        & Banque de l’émetteur & BNP Paribas \tabularnewline
        & Pays de la banque & France \tabularnewline 
        & Notation Pays de la banque & LOW \tabularnewline
        & Résultat du contrôle de la banque & Aucune sanction  \tabularnewline
        & Activité du bénéficiaire & Hébergement et hôtellerie  \tabularnewline 
        \hline
        \multirow{5}{*}{\bf Opération} & Type  & Règlement de facture \tabularnewline
        & Objet  & Frais d'hébergement \tabularnewline
        & Montant& 25000 \tabularnewline
        & Devise & Euros \tabularnewline

        \hline
      \end{tabular}
    \end{scriptsize}
    \caption{Exemple de dossier d'opération conforme \label{tab:tab3}}
  \end{center}
\end{table}

Pour des questions pratiques, nous avons constitué un dictionnaire des
différents secteurs d'activités ainsi que des objets de transactions.
Ce dictionnaire se présente comme suit:
\begin{table}[!h]
 \begin{center}
    \begin{scriptsize}
      \renewcommand{\arraystretch}{2}
      \begin{tabular}{|c|l|}
        \hline
        \rowcolor[gray]{.7}
        \bf \rule[-0.4cm]{0mm}{1cm} Code & \bf libellé du secteur d'activité\\
        \hline

	      Activités extractives & 0 \\
	      Agriculture et chasse & 1 \\
	      Hôtels et restauration & 3 \\
	      Industrie & 4 \\
	      Activites de ménages &5 \\
	      Activités des organisations extraterritoriales &6 \\
	      Activités financières & 7 \\
	      Commerce gros & 8 \\
	      Santé et action sociale & 8 \\
	      Administration publique & 9 \\
	      Transport & 10 \\
	      Education & 11 \\
	      Développements logiciels & 12 \\
	      Maintenance de materiels informatiques & 13
	      Assurance & 14 \\
	      Activités financières & 15 \\
		Fabrication Produits pharmaceutiques & 16 \\
		Construction & 17 \\
		Fabrication de meubles &18 \\
		Activités associatives &19 \\
		Commerce détail & 20 \\
		Fabrication chaussures &21 \\
		Télécommunications & 22 \\
		Activités Juridiques & 23 \\
		Fabrications produits alimentaires & 24 \\
		Service immobilier & 25 \\

		Fabrication de boissons & 26
		Pêche et pisciculture &27\\
        \hline
      \end{tabular}
      \end{scriptsize}
    \caption{Codage des secteurs d'activités}
  \end{table}
