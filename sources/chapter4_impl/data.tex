Dans le tableau \ref{tab:tab3}, nous présentons l'ensemble des caractéristiques de
notre jeu de données. Celles qui sont en italiques représentent les 
caractéristiques d'entrée de notre algorithmes de machine learning.


\begin{table}

  \begin{center}
    \begin{scriptsize}
      \renewcommand{\arraystretch}{2}
      \begin{tabular}{|c|l|l|}
        \hline
        \rowcolor[gray]{.7}
        \bf \rule[-0.4cm]{0mm}{1cm} Section & \bf Caractéristiques & \bf Exemples\\
        \hline
        \multirow{7}{*}{\bf Emetteur de l'ordre} & \textit{Type de personne} & Personne 
        physique 
        \tabularnewline
        & Identité & XXXXXX XXXXXX \tabularnewline 
        & \textit{Résultat du contrôle de l'émetteur} & Aucune sanction  \tabularnewline
        & Pays de résidence & Burkina Faso \tabularnewline
        & \textit{Notation du pays de résidence} & LOW \tabularnewline
        & Banque de l’émetteur & SGBF \tabularnewline
        & Pays de la banque & Burkina Faso \tabularnewline
        & \textit{Notation Pays de la banque} & LOW \tabularnewline
        & \textit{Résultat du contrôle sur la banque} & Aucune sanction  \tabularnewline
        & \textit{Activité de l'emetteur} & Activités extractives \tabularnewline 
        \hline

        \multirow{7}{*}{\bf Bénéficiaire de l'ordre} & \textit{Type de personne} & Personne
        morale 
        \tabularnewline
        & Identité & ZZZZZZZZZZ \tabularnewline
        & \textit{Résultat du contrôle sur la personne} & Aucune sanction  \tabularnewline
        & Pays de résidence & France \tabularnewline 
        & \textit{Notation du pays de résidence} & LOW \tabularnewline
        & Banque de l’émetteur & BNP Paribas \tabularnewline
        & Pays de la banque & France \tabularnewline 
        & \textit{Notation Pays de la banque} & LOW \tabularnewline
        & \textit{Résultat du contrôle de la banque} & Aucune sanction  \tabularnewline
        & \textit{Activité du bénéficiaire} & Hébergement et hôtellerie  \tabularnewline 
        \hline
        \multirow{5}{*}{\bf Opération} & Type  & Règlement de facture \tabularnewline
        & \textit{Objet}  & Frais d'hébergement \tabularnewline
        & \textit{Montant}& 25000 \tabularnewline
        & Devise & Euros \tabularnewline

        \hline
      \end{tabular}
    \end{scriptsize}
    \caption{Exemple de dossier d'opération conforme \label{tab:tab3}}
  \end{center}
\end{table}


Le jeu de données final qui a servi pour l'apprentissage et les tests  est constitué de
six cent (600) dossiers d'opérations.


Pour des questions pratiques, nous avons constitué un dictionnaire des
différents secteurs d'activités ainsi que des objets de transactions.
Un échantillon du dictionnaire est présenté au tableau \ref{tab:codage}.
\begin{table}
 \begin{center}
    \begin{scriptsize}
      \renewcommand{\arraystretch}{2}
      \begin{tabular}{|m{5cm}|m{2cm}||m{5cm}|m{2cm}|}
        \hline
        \rowcolor[gray]{.7}
        \bf \rule[-0.4cm]{0mm}{1cm} Libellé secteur d'activité & \bf Code &
        \bf Libelle du secteur d'activité & \bf Code \\
        \hline

      Activités extractives & 0 &  Activités financières & 15 \\
      Agriculture et chasse & 1 & Hôtels et restauration & 3 \\
      Industrie & 4 & Activites de ménages & 5 \\
      Activités des organisations extraterritoriales & 6 & Activités financières & 7 \\
      Commerce gros & 8 & Santé et action sociale & 8 \\
      Administration publique & 9 & Commerce détail\\
      Transport & 10 & Education & 11 \\
      Développements logiciels & 12 & Maintenance de materiels informatiques & 13\\
      Fabrication Produits pharmaceutiques & 16 & Construction & 17 \\
      Fabrication de meubles & 18 & Activités associatives & 19 \\
      Commerce détail & 20 & Fabrication chaussures & 21 \\
      Télécommunications & 22 & Activités Juridiques & 23 \\
      Fabrications produits alimentaires & 24 & Service immobilier & 25 \\
      Fabrication de boissons & 26 & Pêche et pisciculture & 27 \\
        \hline
      \end{tabular}
      \end{scriptsize}
      \caption{Codage de quelques secteurs d'activités\label{tab:codage}}
  \end{center}
\end{table}
