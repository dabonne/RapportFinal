\chapter{Approche et Implémentation}

\section*{Introduction}
Dans ce chapitre nous présenterons premièrement les données qui ont été utilisés
pour entrainer notre modèle. Ensuite, nous présenterons les résultats obtenus à
la suite de l'utilisation des algorithmes sur nos données. Enfin nous 
présenterons les perspectives envisagées.
Quelle approche est la mieux adaptée à nos besoins? Choisir un algorithme 
d’apprentissage supervisé ou non supervisé dépend habituellement de 
facteurs liés à la structure et au volume de nos données, et le cas 
d’utilisation auquel nous voulons l’appliquer.


\section{Approche}

L'objectif de notre travail est de prouver qu'il est possible de mettre en place un
système permettant de donner l'état d'un dossier à la réception de celui-ci.

Pour cela, nous considérons un dossier comme l'ensemble des informations des 
intervenants et de l'opération elle-même. Ces informations sont relevées 
directement sur le dossier physique. L'idée est de prendre le dossier
et de récupérer tous les éléments d'analyse qui sont présents sur le documents.
Notre dataset initiale nous a été extrait des données du CBS. Il comportait
initialement 10000 lignes.

Nous allons nettoyer nos données en écartant les anomalies, tout en restant
vigilant sur ce que contiennent les données mises de côté. La notion d'anomalie
dans les transferts à l'étranger est très délicate. En effet, ce n'est parce que
M. X a toujours efectué des transferts de 1000 Euros vers un pays A que demain
il n'enverra pas 5 Euros vers le même pays pour le même motif.

La détection d'anomalie dans notre cas n'a pas fourni de résultats satisfaisants
(ils éliminent une énorme partie des données d'entrée , qui ne sont pas 
forcément des anomalie)

Après l'analyse de la conformité bancaire et de son application sur les transferts
à l'étranger, nous avons donc du changer de méthode. L'étude de notre sujet et
des contraintes qui nous sont imposées nous permet de faire un apprentissage
supervisé sur nos données.

Pour une meilleure prédicion, nous posons comme hypothèse de départ que tous
dossiers qui passent par notre modèle est considérés comme complet. Cette
hypothèse nous permet de nous focaliser sur les autres aspects de la conformité.

\section{Choix de la méthode}
Pour notre problème, la méthode qui sera choisie devra nécessairement remplir
ces conditions :
\begin{itemize}
  \item Fonctionner sur un dataset qui qui dispose de très peu de données
  \item Ne pas fonctionner comme une boite noire pour permettre la déduction de
    règles
  \item Acceter des caractéristiques discrètes et continues
  \item Pouvoir prédire un résultat malgré le bruit.
\end{itemize}

Face à ces contraintes notre choix s'est tournée vers les arbres de décisions.

\section{Réalisation}

\subsection{Le jeu de données}
Le jeu de données que nous avons utilisé pour notre modèle a été obtenue à
partir des dossiers physiques mis à notre disposition au niveau de la Société
Générale Burkina Faso. Sur chacun de ces dossier nous récupérons:
\begin{itemize}
  \item Le nom du donneur d'ordre
  \item L'activité du donneur d'ordre
  \item Le pays du donneur d'ordre
  \item Le nom du beneficiaire de l'ordre
  \item L'activité du bénéficiaire de l'ordre
  \item le pays du bénéficiaire de l'ordre
  \item La banque du bénéficiare
  \item Le pays de la banque du bénficiaire
  \item Le motif de l'opération
  \item L'objet de l'opération
  \item Le montant de l'opération
  \item La dévise de l'opération
\end{itemize}
La majeure partie de ces informations est disponible sur le CBS qui consitue
notre principal source de données. D'autres par contre comme l'activité du
bénéficiaire, la banque du bénéficiaire ne sont pas disponibles sur cette base
de données. Pour ces informations, indispensable dans l'analyse d'un dossier de
transfert, nous avons effectué des recherches sur les moteurs de recherches afin
de compléter notre dataset.
Les informations que nous utilisons sont confidentielles et ne peuvent quitter 
le réseau de la SG.
Certaines informations telles que les identités du donneur et du 
bénéficiaire de l'ordre, les pays du donneur et du bénéficiaire, la 
banque du bénéficiaire doivent passer sur des plateformes de vérifications
de sanctions comme force-online. Ces plateformes permettrons de savoir si une
des entités cités ci-dessus est sous sanctions et de quel type de sanctions il
s'agit. Le résultat des différents contrôle entre dans le processus de 
classification d'un dossier de transfert.

En conséquence, les caractéristiques de notre dataset seront:

\begin{itemize}
  \item Le nom du donneur d'ordre
  \item Le résultat du controle de l'identité du donneur
  \item L'activité du donneur d'ordre
  \item Le pays du donneur d'ordre
  \item Le résultat du contrôle du pays du donneur
  \item Le nom du beneficiaire de l'ordre
  \item Le résultat du contrôlede l'identité du bénéficiaire
  \item L'activité du bénéficiaire de l'ordre
  \item le pays du bénéficiaire de l'ordre
  \item Le résultat du contrôle du pays du bénéficiaire
  \item La banque du bénéficiare
  \item Le pays de la banque du bénficiaire
  \item Le motif de l'opération
  \item L'objet de l'opération
  \item Le montant de l'opération
  \item La dévise de l'opération
\end{itemize}
Apres analyse de notre dataset, il ressort que sur la période considéré, nous
avons autant de bénéficiaire que de ligne de données. Ce constat est
pratiquement le même au niveau des identités des donneurs d'ordre. Pour cela et
en vue d'éviter un sur-apprentissage, nous avons opté d'ignorer la colon
Notre jeu de données est constitué de huit cent cinquante(850) dossiers. 

\subsection{Les outils}
Nous allons présenter les outils qui nous ont permis de mettreen place l'outil
que nous avons proposé.

Le langage Javascript: 

Le langage Java

Nous utilisons le langage Python et de nombreuses bibliothèques python pour 
les codes qui serviront à l’implémentation de notre modèle. 

\section{Le système de classification}
  \subsection{Decision Tree}

  \section{Résultats}



ê
