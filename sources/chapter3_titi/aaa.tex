\section{Machine Learning et mise en oeuvre d'un programme de conformité}

Après avoir présenté les composantes d'un programme AML, nous allons analyser
les moyens à mettre en oeuvre au sein des établissements de crédit pour 
appliquer de manière opérationnelle les recommandations du GAFI
\nomenclature{GAFI}{Groupe d'Action FInancière} et surtout 
comment ces moyens pourraient être automatisés grâce au ML.
 
\subsection{Le Machine Learning pour la simplification des procédures KYC}
 
KYC \nomenclature{KYC}{Know Your Customer} est l'acronyme de Know Your Customer. Il désigne le processus permettant 
de vérifier l'identité des intervenants à une opération bancaire afin de 
s'assurer de la conformité des clients face aux législations anti-corruption,
de leur probité et de leur intégrité.

Initialement mise en oeuvre par une intervention humaine, les tâches 
répétitives des procédures KYC pourraient être automatisées grâce au 
Machine learning. Les principales étapes de la procédure KYC qui peuvent être
automaisées par des modèles d'apprentissage automatiques sont:
  
  \subsubsection{L'identification et le contrôle des informations 
    d'identification des clients}
    Il s'agit au cours de cette étape de demander et d'enregistrer les
    informations personnels du clients et de contrôler son identité par rapport
    à une pièce d'identité officielle. A ce niveau, les informations
    personnelles nom, prénoms, date de naissance, situation maritale,
    adresse\ldots dites \og bio data \fg sont demandées.

    Le contrôle des informations d'adresse nécessitera la fourniture par le 
    client d'une pièce probante (facture d'eau, de téléphonie fixe, etc.).
    La banque pourra également adresser un courrier de bienvenue ou de 
    remerciement pour la fidélité au client  et vérifier que le courrier ne
    revient pas.

  \subsubsection{Le contrôle des clients par rapport aux listes de sanctions}
    Lors de toute opération, la banque contrôle la présence éventuelle d'un
     des intervenants de l'opération sur une ou plusieurs listes de 
    sanction, selon la réglementation en vigueur dans le pays. Ces listes sont
    établies par les autorités officielles(nationales ou supranationales comme
    L'ONU, L'Union Européenne). Elles regroupent des individus ou de groupes qui
    compte tenu de leur activités ont été frappé de mesure d'embargo nominative.

  \subsubsection{Qualification du risque de blanchiment}
    Il s'agit là de vérifier si le client n'existe pas sur des listes qui ne 
    sont pas d'ordre public. Ces listes peuvent être celles des PEP
    \nomenclature{PEP}{Personnes Exposées Politiquement} ou une liste d'indésirables car en opposition 
    avec la déontologie et les valeurs du groupe financier.

  \subsubsection{Consignation des pièces d'identification des clients}
    Après la phase d'identification du client et de son contrôle, 
    l'établissement financier doit enregistrer les preuves d'identification du
    client et les archiver.

    
\subsection{Le Machine Learning, un outil essentiel à la détection des 
 transactions suspectes}
 
 La lutte contre le blanchiment d'argent et le financement du terrorisme est
 principalement basée sur l'élaboration, par des algorithmes, de scénarios
 d'anticipation dits « déterministes ». Les algorithmes utilisés analysent
 en temps réel les transactions et sont capables, en quelques instants, de
 décéler une transaction suspecte. Ces scénarios se basent sur des règles
 arrêtées, constantes et ne sont que très peu modifiés une fois mis en 
 place. La procédure est basé sur des mots clés et il est difficile de calibrer
 ces logiciels à un niveau permettant une protection optimale face aux 
 transactions frauduleuses sans générer pour autant un nombre élevé de fausses
 alertes qui de ce fait viendrait perturber les activités de conformité.

 Pour résoudre ces problémes particulièrement chronophages et coûteux pour les
 banques, des applications basées sur le Machine Learning pourraient apprendre
 à identifier les transactions frauduleuses en établissant des procédés 
 standardisés et automatisés. Cela permettra de réduire la charge de travail des
 équipes, tout en affinant la précision de l'analyse. Il s'agirait là pour les
 banques de réduire leurs coùts et leur sanctions, tout en assignant un travail
 à plus forte valeur ajoutée aux équipes chargées de la conformité.
