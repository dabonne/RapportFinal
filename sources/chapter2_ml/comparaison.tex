\section{Comparaison des algorithmes de classification}

Le choix de l'algorithme optimal pour un problème donnée dépend de sa vitesse
d'entrainement et de prédiction, de la précision de ces prévisions, de la
quantité de données nécessaires à l'entrainement, de la facilité à la mettre en
oeuvre, et de la capacité à expliquer le résultat de la prédiction.

Le tableau ci-dessous présente une comparaison des différents algorithmes de
classification.

\begin{table}
  \begin{center}
    \renewcommand{\arraystretch}{1.5}

    \begin{tabular}{|c|c|c|c|c|c|}
      \hline
      \rowcolor[gray]{0.7}
      \bf\rule[-0.4cm]{0mm}{1cm} Algo & \bf Interprétabilite & \bf Précision & \bf{VE et VP} & \bf Données \\
      \hline
     \bf Knn & Oui & Faible & Dépend de K & Beaucoup \\
      \hline
     \bf Régression & Un peu & Faible & Rapide & Peu \\
      \hline
     \bf Naïves bayes & Un peu & Faible & Rapide & Peu \\
      \hline
     \bf Réseaux de neurones & non & Très élevé & Lent & Beaucoup \\
      \hline
      \bf{Arbre de décision} & Oui & Moyen & Rapide & Assez \\
      \hline
      \bf{Random Forest} & Non & Très élevé & Lent & Assez \\
      \hline
    \end{tabular}
    \caption{Tableau de comparaison des algorithmes de classifications}
    \label{tab:tab2}
  \end{center}
\end{table}
Le tableau \ref{tab:tab2} révèle que les algorithmes de Réseaux de neurones et ceux de forêt
aléatoire ont un taux très élevé de bonne prédiction. Malheureusement, ces
algorithmes fonctionnent bien sur des jeude données énormes. De plus la vitesse
d'apprentissage et de prédiction reste relativement lente par rapport aux
autres algorithmes.

