\chapter{Contexte général de l'étude}

\section*{Introduction}
L’un des critères les plus importants de la réussite d’un projet est la 
satisfaction du client et des utilisateurs finaux. Nous ne pouvons satisfaire
le demandeur sans avoir compris les problèmes qui ont suscité la naissance du projet.
Dans ce chapitre, nous présenterons dans un premier temps la structure dans
laquelle nous avons effectué notre stage. En second lieu, mous parlerons du
projet qui nous a été confié, de son intéret pour l'entreprise, de ses
objectifs et de la problématique y afférente.

\section{La structure d'accueil}

\subsection{Présentation et Histoire}

 \subsubsection{Présentation de la Société Générale Burkina Faso}

 Nous avons effectué notre stage au sein de la Société Générale Burkina Faso \nomenclature{SGBF}{Société Générale Burkina Faso}
une filiale du groupe français \nomenclature{SG}{Société Générale} Société
Générale.
La Société Générale Burkina Faso exerce dans la banque de détail et les services
financiers, la gestion d’actifs et services aux investisseurs et dans la banque
de financement et d’investissement. Elle a pour ambition\footnote{https://societegenerale.bf/fr/votre-banque/presentation}
de:
\begin{itemize}
  \item Bâtir avec ses clients une relation équilibrée et équitable où elle est
    est avec eux , à leurs côtés pour les aider à progresser;
  \item Mettre sa performance au service de ses clients;
  \item Etre la banque relationnelle, référence sur ses marchés, choisie pour la
    qualité et l'engagementde ses équipes;
\end{itemize}

\subsubsection{Historique de la SGBF}

La SGBF a été créée en mai 1998 avec la participation de l'Etat Burkinabé et de
plusieurs partenaires financiers nationaux et internationaux. Elle est née de la 
cession par l’état de $51\%$ du capital de la Banque pour le Financement du Commerce et des
investissements au Burkina (BFCI-B)
\footnote{https://societegenerale.bf/fr/votre-banque/presentation/notre-histoire}.


  \begin{description}
     \item[Septembre 1973:] Création de la Caisse Nationale des Dépots et des
       Investissements(CNDI)\nomenclature{CNDI}{Caisse Autonome des Dépots et des
       Investissements}.
    \item[Août 1984:] Création de l'Union Révolutionnaire de
      Banques(UREBA)\nomenclature{UREBA}{Union Révolutionnaire de Banques}
    \item[Juin 1986:] Création de la Caisse Autonome
      d'Investissement(CAI).\nomenclature{CAI}{Caisse Autonome d'Investissement}
     \item[Août 1986:] Transformation de la CNDI en banque commerciale sous la forme
  d s'une société d'économie mixte.
    \item[Décembre 1987:] Changement de dénomination de CNDI en BFCI-B
    \item[Février 1991 à Décembre 1996:] Mise sous administration provisoire du
  Groupe BFCI-BUREBA-CAI. Fusion-absorption de l'UREBA et de la CAI par 
  la BFCI-B en mai 1995.
    \item[Février 1997:] Cession par l'état de 34\% du capital à des privés
      nationaux.
    \item[Mai 1998:] Cession par l'état de 51\% du capital à des partenaires
      étrangers. La BFCI-B devient la Société Générale des Banques du
      Burkina(SGBB).
    \item[08 février 2013:] Changement de dénomination sociale: la SGBB devient
      la Société Générale Burkina Faso(SGBF).
  \end{description}

  \subsection{Organisation}
   \subsubsection{Organigramme}  
L'organnigramme de la Société Générale Burkina Faso se présente comme suit: 
(voir Figure \ref{fig:organigramme}).
 \begin{figure}[h!]
  \begin{center}
    \includegraphics[width=17cm]{images/organigramme.png}
\caption{Organigramme de la société générale Burkina
Faso.\label{fig:organigramme}}
\end{center}
\end{figure}


  \subsubsection{La direction des resources}

Nous avons effectué notre stage au sein de la Direction des Ressources plus précisément
dans la cellule Innovation de cette direction. La Direction des Ressources est 
l’entité chargée de gérer toutes les ressources matérielles et logicielles de la
banque. Les missions de la cellule Innovation, service dans lequel nous avons 
effectué notre stage sont les suivantes:
\begin{itemize}
  \item Continuer de diffuser la culture de l’innovation
  \item Identifier de nouveaux business et services pour les clients
  \item Développer de nouveaux process optimaux dans la réalisation des tâches quotidiennes
    des collaborateurs.
  \item Favoriser l’émergence d’innovations de rupture et tirer parti des technologies et de
    la gestion des données.
\end{itemize}
 
 \section{Présentation du sujet}

\subsection{Libellé du sujet}

\textbf{Application des techniques de Machine Learning à l’analyse de la 
conformité des dossiers de transferts à l'étranger.}

 \subsection{Contexte du sujet}

 L’esprit d'équipe, l’innovation, la responsabilité et l'engagement 
 traversent toutes les activités de la banque.  A travers cette approche,
 la Société Générale cherche des moyens qui lui permettront d’améliorer
 la satisfaction client par un traitement qualitatif et rapide des 
 opérations qui leur ont été soumis. 
 
 Dans ses missions quotidiennes, la SGBF effectue pour ses clients des 
 opérations à l’étranger(pays n'appartenant pas à la zone UEMOA). Ces 
 opérations comportent de nombreux risques de violation des différentes 
 règlementations en cours.

Effectuer une opération de transfert à l'étranger commence par la constitution
d'un dossier appellé dossier de transfert.
Lorsqu’un client veut effectuer un transfert vers un pays  étranger, la
règlementation exige qu’il transmette à l’intermédiaire agréé qui est la banque
un certain nombre d’éléments permettant de justifier le transfert qu’il voudrait
initier. L'ensemble de ces éléments constitue le dossier de transfert. L’analyse
du dossier par les experts de la règlementation s’effectue après que le client 
soit reparti. Une journée complète peut s'écouler entre le moment où le dossier
est déposé au guichet et celui où le transfert est réellement effectuer et ceci 
lorsque le dossier ne comporte aucune irrégularité. 

En cas d’irrégularités, le dossier est transmis au conseiller du client qui se 
chargera de le contacter pour lui demander de passer corriger les irrégularités
qui ont été constatées sur son dossier. Lorsque les irrégularités ont été
corrigées, le dossier reprend le circuit d'analyse.

Dans un domaine où la satisfaction du client et la célérité dans le 
traitement des opérations est une exigence et face à l'augmentation 
significative des règlementations et obligations de communications qui en
découlent, vingt-quatre heures pour traiter une opération est un délai déjà
long. Lorsque le dossier comporte des irrégulariés ce délai sera encore
rallonger d'au moins 24 heures encore. 

Lorsque un dossier non conforme à la conformité passe malgré le système d'analyse
mis en place par l'institution financière en l'occurance la SGBF ce sont 
d'importantes sanctions financières et disciplinaires qui lui seront imposées.
En 2018 par exemple, la SG a payé une amende d'environ 1.2 milliards d'Euros 
pour des opérations en dollars vers des entités sanctionnées par les autorités 
américaine.

Au vue des enjeux économiques que suscitent ces situations, nous nous sommes 
posés la question comment permettre aux collaborateurs du services Opérations
Internationales(OPI) \footnote{Le service OPI est un service appartenant à la
direction des Opérations(DOPE)}
\nomenclature{OPI}{OPérations Internationales} de donner à un client l’état de son opération
vis à vis de la règlementation sous réserve d'un contrôle beaucoup plus approfondi. 
Cela permettra d'une part de diminuer la charge de travail des collaborateurs de la banque
et rassurera d'autre part le client sur la volonté des collaborateurs de 
traiter diligemment de son opération.

\subsection{Intérêt du sujet}
 
Ce projet est d’un intérêt très élevé pour la SGBF car elle vise à
apporter des solutions pour le traitement rapide des opérations à l'étranger.
Les résultats attendus sont:
\begin{itemize}
  \item Un gain en temps,
  \item Une facilité de prise de décision,
  \item Une grande disponibilité et de performance 
  \item Un système toujours actuel et compatible
  \item Faciliter la gestion des réclamations
\end{itemize}

    \subsection{Problématique du sujet}

Le service des opérations internationales (OPI) reçoit en moyenne quatre-vingt-dix
(90) dossiers d’opérations de transferts par jour. Ces opérations sont de
plusieurs types et comprennent transactions quotidiennes, périodiques et 
apériodiques effectuées par, ou touchant à de nombreuses parties prenantes
telles que les employés, les clients, les débiteurs et des entités externes.
La nature complexe de ces activités et de ces activités et transactions 
nécessite une surveillance constante pour s’assurer que ni la banque, ni
ses employés ne sont exposés à des risques.

Une analyse rigoureuse est donc appliquée après réception d’un dossier au guichet.
Cette analyse concerne aussi bien les différents intervenants de l'opération que
la nature l'opération elle-même.
Cette analyse peut être décrite suivant trois axes majeurs :
\begin{itemize}
  \item Analyse de la complétude du dossier 
  \item Analyse de la fiabilité des différents intervenants de l'opération 
  \item Analyse de la conformité règlementaire de l'opération
\end{itemize}
C'est après toutes ces analyses que le dossier est transmis pour saisie et
validation dans  le CBS\nomenclature{CBS}{Core Banking System}. Si le dossier comporte des 
irrégularités, il est transmis au conseiller ou à la conformité pour demande
d'accord. Sinon la saisie est validée et le transfert éffectué.

Pour pouvoir mettre en place un système qui puisse analyser les dossiers de ces
opérations, il nous faudra trouver des réponses aux questions suivantes:
\begin{itemize}
  \item Comment codifier le dossier d'une opération à l'étranger ?
  \item Quels sont les caractéristiques qui entre en jeux dans l'analyse d'un
    dossier de transfert ?
  \item La règlementation financière est un outil  qui évolue. Comment mettre en
    place un système qui puisse très vite s'adapter à des évolutions ?
  \item La Banque a l'obligation de communiquer avec le client sur la raison
    pour laquelle son opération a été rejetée au cas où elle l'est. Quels 
    méthodes d'apprentissage sied le mieux à ce problème ?
\end{itemize}
Pour répondre à ces interrogations, nous allons explorer tous les contours du
domaine d'étude, afin de fixer les bases solides qui nous permettront de mener
notre projet à son terme.

\subsection{Objectifs}

L’objectif de l’étude est de développer un environnement permettant une analyse des dossiers et des 
intervenants des différentes opérations de transferts.
Spécifiquement, il s'agit de 
\begin{itemize}
\item comprendre le processus d'analyse de dossier afin de dégager les grandes étapes;
\item Proposer un modèle de machine learning permettant d'analyser un dossier de
  transfert fourni en paramètre vis à vis de la règlementation;
\item Utiliser le modèle qui aura été mis en place afin de détecter des
  transactions suspectes;
\item Mettre en place une interface web permettant d'utiliser ce modèle.
\end{itemize}

\section{Concepts clés du sujet}

\subsection{Vocabulaire}

La compréhension de certains concepts est indispensable à la compréhension du 
sujet.

\subsubsection{Dossier de Transfert}

Un dossier de transfert est l'ensemble des documents fournis par un client 
dans le but de l'exécution d'une opération à l'étranger. Ces documents sont de 
plusieurs types et sont principalement composés de:

\begin{description}
  \item{\textbf{Un ordre de virement :}} Il est donné par le propriétaire d'un compte
    bancaire qui doit payer une prestation ou un créancier ou faire un 
    transfert. L'ordre de virement demande à la banque de débiter une somme de 
    son compte pour créditer un autre compte. Le compte à créditer peut se 
    trouver dans la même banque ou dans une autre banque. Ce document est
    obligatoire dans la réalisation d'une opération
   
  \item{\textbf{Une autorisation de change :}} Il s'agit d'un document obligatoire dans 
  la constitution d'un dossier de transfert à l'étranger. Ces opérations
    s'effectuant en dévise, l'autorisation de change autorise le change vers la
    dévise dans laquelle le transfert sera effectué.
    
  \item{\textbf{La délaration préalable d'importation :}} La Déclaration Préalable 
    d’Importation (DPI) est une formalité accomplie  au sein du ministère en 
    charge du commerce préalablement à toute opération d’importation de 
    marchandises dont la valeur FOB est supérieure ou égale à 500 000 FCFA.
    
  \item{\textbf{L'autorisation spéciale d'importation ou d'exportation :}} Ces documents 
    concernent des produits dont la liste est fixée par avis ministériel. De ces 
    produits, nous pouvons citer le sésame, les céréales, les amande de Karité,
    le sucre\ldots
 
  \item{\textbf{Les documents justifiant l'opération :}} Il s'agit pour des achats de
    marchandise des factures par exemple, pour une inscription dans une école de
    l'attestation d'inscription et du passeport du concerné, pour le règlement 
    d'un salaire du contrat de travail et du bulletin de paie \ldots. 
\end{description} 
   
   Les documents entrants en compte dans la constitution d'un dossier sont 
   nombreux et les éléments cités ci-dessus sont loin d'être exhaustifs.  

L'analyse d'une opération à l'étranger revient à analyser l'ensemble des 
informations contenues sur chacun de ces documents.

\subsubsection{Pays étranger}

Le terme étranger désigne tous les pays en dehors de l’UEMOA. Les 
transferts dans ces pays s'effectue en dévise.

Selon la terminologie du règlement 09/2010/CM/UEMOA relatif aux relations 
financières extérieurs des états membres de l'UEMOA, 
\begin{quote}
 \textit{Le terme étranger désigne  tous les pays en dehors de l'UEMOA pour le 
 contrôle de la position des établissements de crédit vis-à-vis de 
 l'étranger aainsi que pour le traitement des opérations suivantes: 
 domiciliation des exportations sur l'étranger et rapatriement du produit 
 de leur recettes, émission et mise en vente de valeurs mobilières 
 étrangères, importation et exportation d'or, opération d'investissement
 et d'emprunt avec l'étranger, exportation matérielle des moyens de 
 paiement et de valeurs mobilières par colis postaux ou envois par la 
 poste.}
\end{quote}

\subsubsection{Résidents et Non-Résidents dans un Etat}

Sont considérés comme résidents les personnes physiques ayant leur résidence
habituelle dans l'Etat considéré. Sont considérés comme non-résidents ayant leur
résidence habituelles à l'étranger.


Les opérations à l'étranger sont nombreuses et pour chaque type d'opération, les 
intervenants ou acteurs de la transaction sont différents.

\subsection{ Les diférentes opérations à l'étranger}

Plusieurs types d'opérations sont effectuées par le service des opérations 
internationales de la SGBF.

\subsubsection{Les transferts émis}

Il s'agit d'opérations émises  par la banque résidente en l'occurence la SGBF
à destination d'une autre banque présente dans un autre pays. On distingue 
quatres intervenants dans une opération de transfert émis:
\begin{itemize}
  \item L'émetteur de l'ordre qui est le donneur d'ordre
  \item La banque domiciliatrice de l'émetteur en l'occurence dans notre cas la
    SGBF.
  \item La banque du bénéficiaire de l'ordre
  \item Le Bénéficiaire de l'ordre
\end{itemize}

Dans le cas de la SGBF et de toutes les filiales SG, il existe des hubs en 
l'occurance SG New York pour les opérations en dollars et SG Paris pour toute 
les autres dévises. Les différents ordres sont envoyés vers ces hubs qui sont 
chargés de les acheminer vers les différentes banques bénéficiaires.   

 \subsubsection{Les transferts reçus}
   Par transfert reçu, on entend tout virement en provenance de l'étranger à
   destination d'une banque résidente. Les transferts reçus s'effectue par 
   transmission à la banque réceptrice d'un message SWIFT. Comme dans le cas des
   transferts émis nous distinguons quatres intervenants dans cette opération.
   
 \subsubsection{Les opérations de crédit documentaire}
   Le Crédit Documentaire est l’opération par laquelle une banque s’engage, à la
   demande et pour le compte de son client importateur, à régler à un tiers 
   exportateur, dans un délai déterminé, un certain montant contre remise des 
   documents strictement conformes et cohérents entre eux, justifiant de la 
   valeur et de l’expédition des marchandises ou des prestations de services.
   On distingue quatre intervenants pour assurer la sécurité de l'opération:
   \begin{itemize}
     \item L'Acheteur/Importateur = Donneur d'ordre
     \item La Banque de l'Acheteur = Banque Emettrice
     \item La Banque du vendeur = Banque notificatrice et/ou Banque confirmatrice
     \item Le vendeur/L'Exportateur = Bénéficiaire
   \end{itemize}
   \begin{figure}[h!]
     \begin{center}
       \includegraphics[width=12cm]{images/credit_doc.png}
        \caption{Circuit d'une opération de crédit documentaire.
        \label{fig:credit}}
     \end{center}
   \end{figure}
   
 \subsubsection{Les opérations de remise documentaire}
 
  La remise documentaire consiste pour le vendeur à faire encaisser par une 
  banque le montant dû par un acheteur contre remise de documents. Les
  documents sont remis à l'acheteur uniquement contre paiement ou acceptation
  d'une lettre de change.
  Les intervenants dans l'opération d'encaissement sont :
  \begin{itemize}
    \item Le Donneur d'ordre (le client)
    \item La Banque remettante (la banque du client)
    \item La banque chargée de l'encaissement (autre banque que la banque remettante)
    \item La Banque présentatrice (banque chargée de l'encaissement)
  \end{itemize}
  \begin{figure}[h!]
    \begin{center}
      \includegraphics[width=12cm]{images/remise_doc.png}
         \caption{Circuit d'une opération de remise documentaire.
         \label{fig:remise}}
    \end{center}
   \end{figure}
   
   
 \subsubsection{Les remises de chèques hors UEMOA}
 
  La remise de chèques correspond au dépôt d'un ou de plusieurs chèques par un 
  client auprès de sa banque afin que celle-ci en assure le recouvrement. Chaque
  chèque remis doit être signé au dos par le client bénéficiaire à qui, la 
  banque demande, le plus souvent, d'indiquer le numéro de compte à créditer au 
  dos du chèque.




\section*{Conclusion}
Il  a été question dans ce chapitre de présenter  la structure d’accueil,
la SGBF, qui a suivi et coordonné tous les travaux .Ensuite nous avons
présenté le sujet qui nous a été confié, dégager sa problématique et
l’intérêt qu’il suscite pour la Société Générale Burkina Faso. Dans la suite,
il sera question  d’aborder le machine learning et les algorithmes de classification.
