\chapter{Contexte général de l'étude}

\section*{Introduction}
Ce stage resulte de plusieurs collaborations. D’une part elle result d’une collabo-
ration entre l’Université Nazi Bonin et L’université Lumière Lyon 2 dans le cadre du
programme erasmus+ mobilité etudiantes. C’est dans ce cadre que nous avons été
acceuillit au dans les locaux du LIRIS au sein de l’Université Lumière Lyon 2. Ce su-
jet de stage intitulé «segmentation géométrique et ghotométrique d’images acquises
par drones», resulte aussi d’une collaboration entre la société TECNI DRONE basée
à Baix (Ardèche) et l’equipe Imagine du LIRIS (site de l’Université Lumière Lyon
2 à Bron). TECNI DRONE se specialise dans la formation de pilot de drones et dans
l’acquisition de données géométriques issues de mines et de carrières. Les campagnes
d’acquisition d’images par drones, en milieu naturel, permettent de produire avec un
très haut niveau de qualité, des modèles numériques de terrains texturés, porteurs à
la fois d’informations géométriques et d’informations photométriques extrêmement
riches. De nombreuses recherches ont été menées au cours des dernières années, pour
affiner la qualité du traitement de ces données visuelles, et produire des maillages
texturés de plus en plus fiables. En revanche, l’exploitation de ces données reste pour
l’instant limitée soit à des traitements géométriques effectués sur les maillages 3D,
soit à des traitements d’images 2D, effectués par exemple sur les ortho-images obtenues 
par assemblage des multiples vues partielles.

\section{La structure d'accueil}

\subsection{Présentation }

 \subsubsection{Présentation du LIRIS}

Nous avons effectué notre stage au sein du Laboratoire d'InfoRmatique en 
Image et Systeme d'information \nomenclature{LIRIS}{Laboratoire d'InfoRmatique en Image et Systeme d'information}.
Le \nomenclature{LIRIS}{Laboratoire d'InfoRmatique en Image et Systeme d'information}est une unité miste de recherche  (\nomenclature{UMR}{Unité Mixte de Recherche} 5205) porté par
\begin{itemize}
  \item le CNRS
  \item l'INSA de Lyon
  \item l'Université Claude Bernard Lyon 1
  \item l'Université Lumière Lyon 2
  \item l'Ecole Centrale de Lyon
\end{itemize}
Il compte 330 membres, et a pour principal champ scientifique 
l'Informatique et plus généralement les Sciences et Technologies de l’Information.
Une partie importante de la recherche effectuée au LIRIS s’étend à la frontière de notre 
discipline, au service de problématiques sociétales importantes. Certaines des ses activités de 
recherche se situent aux interfaces de l’ingénierie, des sciences humaines et sociales, des sciences 
de la vie et des sciences de l’environnement. L’ensemble des 6 pôles de compétences du LIRIS participe 
de façon équilibrée à la valorisation des travaux de recherche. Par ailleurs, le LIRIS entretient de 
nombreuses relations avec son environnement social, économique et culturel, aussi bien aux niveaux 
local et régional qu’au niveau  national. 
Les interactions avec les entreprises s’établissent au travers de projets collaboratifs.
Le LIRIS couvre des thématiques scientifiques structurées en 6 pôles de compétences regroupant 14 équipes. 
\begin{itemize}
  \item Simulation, virtualité \& sciences computationnelles (Equipes Beagle, R3AM, SAARA)
  \item Géométrie \& modélisation (Equipes GeoMod, M2DisCo)
  \item Science des données (Equipes BD, DM2L, GOAL)
  \item Vision intelligente \& reconnaissance visuelle (Equipe Imagine)
  \item Interactions \& cognition (Equipes SICAL, SMA, TWEAK)
  \item Services, systèmes distribués, sécurité (Equipes DRIM, SOC)
\end{itemize}

Les travaux des équipes de recherche trouvent aussi des applications dans les secteurs : 
Biologie et santé (modelisation du vivant, ingénierie pour la santé), Intelligence ambiante 
(systèmes pervasifs et distribués, monitoring intelligent, systèmes autonomes), Apprentissage
humain (personnalisation, assistance cognitive, assistance à l'apprentissage collaboratif, 
jeux sérieux, loisirs numériques), Calcul scientifique (traitement de grandes masses de données
– big data).


\subsubsection{Presentation de l'equipe Imagine}
Nous avons effectués notre stage au sein de l'equipe Imagine sur le site de l'
université Lumière Lyon 2 à Bron.
L’équipe Imagine du LIRIS est spécialisée dans la vision par ordinateur, l’apprentissage 
et la reconnaissance de formes. Elle réunit 21 membres permanents (8 PR, 3 MCF-HDR et 10 MCF), 
enseignants-chercheurs de l’Université Lyon 1, de l’Université Lyon 2, de l’École Centrale de 
Lyon et de l’INSA Lyon.
En 2019, elle compte également parmi ses membres 29 doctorants et 17 post-doctorants.
Les différentes activités de recherche menées dans l’équipe Imagine partagent les mêmes 
objectifs généraux visant la compréhension d’images multi-sources et multi-capteurs, 
intégrant ainsi une très large variété de contenus autour des images de personnes, d’objets 
et de scènes en 2D et 3D (scènes naturelles ou urbaines, images aériennes et satellites, visages…), 
des séquences d’images et des flux vidéos, ainsi que des documents numérisés (cartes, textes écrits et 
imprimés, partitions et symboles…). La notion d’objet visuel, au sens large, constitue ainsi un 
dénominateur commun de nos recherches.
Les activités de l’équipe Imagine se déclinent en différentes thématiques 
liées à la mise en œuvre de méthodes d’indexation, de modélisation, 
de classification et de reconnaissance du contenu (objets, actions, concepts), 
avec une attention particulière portée au développement de méthodes d’apprentissage 
automatique pour la vision par ordinateur.
Les recherches menées dans l’équipe Imagine visent à construire des passerelles pour 
franchir le fossé sémantique entre, d’une part, les informations de bas niveau présentes 
dans les données brutes (données échantillonnées issues du signal), les éventuelles données 
multi-sources issues d’autres modalités ou capteurs (cas notamment des applications embarquées) 
et les informations de plus haut niveau sémantique qui reposent sur la modélisation, la 
classification et l’identification des contenus.
L’activité de recherche de l’équipe Imagine relève de 5 
sous-thèmes majeurs qui constituent le cœur de ses applications.

\includegraphics[scale=0.4]{images/Imagine.png}~\\[0.8cm]


 

  \subsection{Organisation}
   \subsubsection{Organigramme}  
L'organnigramme du LIRIS se présente comme suit: 
(voir Figure \ref{fig:organigramme}).
 \begin{figure}[h!]
  \begin{center}
    \includegraphics[width=17cm]{images/OrganigrammeLiris.png}
\caption{Organigramme du LIRIS.\label{fig:organigramme}}
\end{center}
\end{figure}


 \section{Présentation du sujet}

\subsection{Intitulé de sujet}

\textbf{Segmentation Géométrique etPhotométrique d’images Acquises par Drones}

 \subsection{Contexte du sujet}



\subsection{Intérêt du sujet}
 
Ce projet est très important pour Technidrone dans la mesure ou elle vise à
assister les techniciens de Technidrone dans leur tache d'annotation par la meme occasions reduire de façon 
significative le temps de travail de ces derniers.
Les résultats attendus sont:
\begin{itemize}
  \item Un gain en temps;
  \item Identification automatique des formes geometriques dans les images;
  \item 
\end{itemize}

    \subsection{Problématique du sujet}

    L’exploitation de données recoltées lors des campagnes d'acquisition reste pour
    l’instant limitée soit à des traitements géométriques effectués sur les maillages 3D,
    soit à des traitements d’images 2D, effectués par exemple sur les ortho-images obtenues
    par assemblage des multiples vues partielles. En revanche, la qualification de
    ces données, indispensable aux exploitants, reste pour l’instant un travail essentiellement manuel.
    Il s’agit par exemple, d’identifier les ruptures de pentes qui délimitent
    les voies de roulement des engins, de calculer la largeur de ces voies, de délimiter
    les hauts et les bas de talus, de calculer le volume des tas correspondant aux matériaux 
    extraits des carrières, etc. Ce processus qui est très important pour créer la
    carte d’une mine ou d’une carrière prend une dizaine d’heures pour un personnel
    entraîné. Une première etude effectuée au sein du laboratoire, axée essentiellement
    sur la geometrie contenue dans les maillages 3D a permis d’obtenir 62\% de rappel
    et 10\% de precision.

\subsection{Objectifs}

L’objectif de stage est d’utiliser d'une part les informations issues de la texture pour 
ameliorer les resultats obtenus en se basant uniquement sur la geometrie, et d'autre part, d’utiliser
de manière conjointe des informations issues des maillages, 
décrivant la géométrie de la scène, avec des données issues de la texture, portant des informations 
sur les discontinuités photométriques du terrain, pour assister les opérateurs dans leur tâche d’annotation 
des modèles numériques, construits à partir des images acquises par les drones.
En se basant sur des terrains déjà annotés, et en entraînant des classifieurs à reconnaître
les structures géométriques ou les motifs d’intérêt, nous voulons évaluer la capacité d’un système automatisé 
à effectuer cette identification avec un taux de succès le plus élevé possible. La tâche de 
l’opérateur se limiterait alors à la correction des inévitables erreurs de classification.

\section{Concepts clés du sujet}

\subsection{Vocabulaire}

La compréhension de certains concepts est indispensable à la compréhension du 
sujet.

\subsubsection{Dossier de Transfert}

Un dossier de transfert est l'ensemble des documents fournis par un client 
dans le but de l'exécution d'une opération à l'étranger. Ces documents sont de 
plusieurs types et sont principalement composés de:

\begin{description}
  \item{\textbf{Un ordre de virement :}} Il est donné par le propriétaire d'un compte
    bancaire qui doit payer une prestation ou un créancier ou faire un 
    transfert. L'ordre de virement demande à la banque de débiter une somme de 
    son compte pour créditer un autre compte. Le compte à créditer peut se 
    trouver dans la même banque ou dans une autre banque. Ce document est
    obligatoire dans la réalisation d'une opération
   
  \item{\textbf{Une autorisation de change :}} Il s'agit d'un document obligatoire dans 
  la constitution d'un dossier de transfert à l'étranger. Ces opérations
    s'effectuant en dévise, l'autorisation de change autorise le change vers la
    dévise dans laquelle le transfert sera effectué.
    
  \item{\textbf{La délaration préalable d'importation :}} La Déclaration Préalable 
    d’Importation (DPI) est une formalité accomplie  au sein du ministère en 
    charge du commerce préalablement à toute opération d’importation de 
    marchandises dont la valeur FOB est supérieure ou égale à 500 000 FCFA.
    
  \item{\textbf{L'autorisation spéciale d'importation ou d'exportation :}} Ces documents 
    concernent des produits dont la liste est fixée par avis ministériel. De ces 
    produits, nous pouvons citer le sésame, les céréales, les amande de Karité,
    le sucre\ldots
 
  \item{\textbf{Les documents justifiant l'opération :}} Il s'agit pour des achats de
    marchandise des factures par exemple, pour une inscription dans une école de
    l'attestation d'inscription et du passeport du concerné, pour le règlement 
    d'un salaire du contrat de travail et du bulletin de paie \ldots. 
\end{description} 
   
   Les documents entrants en compte dans la constitution d'un dossier sont 
   nombreux et les éléments cités ci-dessus sont loin d'être exhaustifs.  

L'analyse d'une opération à l'étranger revient à analyser l'ensemble des 
informations contenues sur chacun de ces documents.

\subsubsection{Pays étranger}

Le terme étranger désigne tous les pays en dehors de l’UEMOA. Les 
transferts dans ces pays s'effectue en dévise.

Selon la terminologie du règlement 09/2010/CM/UEMOA relatif aux relations 
financières extérieurs des états membres de l'UEMOA, 
\begin{quote}
 \textit{Le terme étranger désigne  tous les pays en dehors de l'UEMOA pour le 
 contrôle de la position des établissements de crédit vis-à-vis de 
 l'étranger aainsi que pour le traitement des opérations suivantes: 
 domiciliation des exportations sur l'étranger et rapatriement du produit 
 de leur recettes, émission et mise en vente de valeurs mobilières 
 étrangères, importation et exportation d'or, opération d'investissement
 et d'emprunt avec l'étranger, exportation matérielle des moyens de 
 paiement et de valeurs mobilières par colis postaux ou envois par la 
 poste.}
\end{quote}

\subsubsection{Résidents et Non-Résidents dans un Etat}

Sont considérés comme résidents les personnes physiques ayant leur résidence
habituelle dans l'Etat considéré. Sont considérés comme non-résidents ayant leur
résidence habituelles à l'étranger.


Les opérations à l'étranger sont nombreuses et pour chaque type d'opération, les 
intervenants ou acteurs de la transaction sont différents.

\subsection{ Les diférentes opérations à l'étranger}

Plusieurs types d'opérations sont effectuées par le service des opérations 
internationales de la SGBF.

\subsubsection{Les transferts émis}

Il s'agit d'opérations émises  par la banque résidente en l'occurence la SGBF
à destination d'une autre banque présente dans un autre pays. On distingue 
quatres intervenants dans une opération de transfert émis:
\begin{itemize}
  \item L'émetteur de l'ordre qui est le donneur d'ordre
  \item La banque domiciliatrice de l'émetteur en l'occurence dans notre cas la
    SGBF.
  \item La banque du bénéficiaire de l'ordre
  \item Le Bénéficiaire de l'ordre
\end{itemize}

Dans le cas de la SGBF et de toutes les filiales SG, il existe des hubs en 
l'occurance SG New York pour les opérations en dollars et SG Paris pour toute 
les autres dévises. Les différents ordres sont envoyés vers ces hubs qui sont 
chargés de les acheminer vers les différentes banques bénéficiaires.   

 \subsubsection{Les transferts reçus}
   Par transfert reçu, on entend tout virement en provenance de l'étranger à
   destination d'une banque résidente. Les transferts reçus s'effectue par 
   transmission à la banque réceptrice d'un message SWIFT. Comme dans le cas des
   transferts émis nous distinguons quatres intervenants dans cette opération.
   
 \subsubsection{Les opérations de crédit documentaire}
   Le Crédit Documentaire est l’opération par laquelle une banque s’engage, à la
   demande et pour le compte de son client importateur, à régler à un tiers 
   exportateur, dans un délai déterminé, un certain montant contre remise des 
   documents strictement conformes et cohérents entre eux, justifiant de la 
   valeur et de l’expédition des marchandises ou des prestations de services.
   On distingue quatre intervenants pour assurer la sécurité de l'opération:
   \begin{itemize}
     \item L'Acheteur/Importateur = Donneur d'ordre
     \item La Banque de l'Acheteur = Banque Emettrice
     \item La Banque du vendeur = Banque notificatrice et/ou Banque confirmatrice
     \item Le vendeur/L'Exportateur = Bénéficiaire
   \end{itemize}
   \begin{figure}[h!]
     \begin{center}
       \includegraphics[width=12cm]{images/credit_doc.png}
        \caption{Circuit d'une opération de crédit documentaire.
        \label{fig:credit}}
     \end{center}
   \end{figure}
   
 \subsubsection{Les opérations de remise documentaire}
 
  La remise documentaire consiste pour le vendeur à faire encaisser par une 
  banque le montant dû par un acheteur contre remise de documents. Les
  documents sont remis à l'acheteur uniquement contre paiement ou acceptation
  d'une lettre de change.
  Les intervenants dans l'opération d'encaissement sont :
  \begin{itemize}
    \item Le Donneur d'ordre (le client)
    \item La Banque remettante (la banque du client)
    \item La banque chargée de l'encaissement (autre banque que la banque remettante)
    \item La Banque présentatrice (banque chargée de l'encaissement)
  \end{itemize}
  \begin{figure}[h!]
    \begin{center}
      \includegraphics[width=12cm]{images/remise_doc.png}
         \caption{Circuit d'une opération de remise documentaire.
         \label{fig:remise}}
    \end{center}
   \end{figure}
   
   
 \subsubsection{Les remises de chèques hors UEMOA}
 
  La remise de chèques correspond au dépôt d'un ou de plusieurs chèques par un 
  client auprès de sa banque afin que celle-ci en assure le recouvrement. Chaque
  chèque remis doit être signé au dos par le client bénéficiaire à qui, la 
  banque demande, le plus souvent, d'indiquer le numéro de compte à créditer au 
  dos du chèque.




\section*{Conclusion}
Dans ce chapitre il  a été question dans ce chapitre de présenter  la structure d’accueil,
au sein de laquelle nous avons menés nos travaux de recherche. .Ensuite nous avons
présenté le sujet qui qui fait objet de ce memoire, dégager sa problématique et
l’intérêt qu’il suscite pour la Société Technidrone. Dans le chapitre suivant,
nous parlerons des techniques machine learning, de la segmentation et de la classification.
